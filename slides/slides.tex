\documentclass[aspectratio=169]{beamer}
\usetheme{Singapore}

% add page numbers at the bottom of the slides
\setbeamertemplate{caption}[numbered]
\addtobeamertemplate{navigation symbols}{}{%
    \usebeamerfont{footline}%
    \usebeamercolor[fg]{footline}%
    \hspace{1em}%
    \raisebox{1.4pt}[0pt][0pt]{\insertframenumber/\inserttotalframenumber}
}

% \definecolor{primarycolor}{HTML}{0000FF}

% \makeatletter

% \def\sectioncolor{primarycolor}% color to be applied to section headers

% \setbeamercolor{palette primary}{use=structure,fg=structure.fg}
% \setbeamercolor{palette secondary}{use=structure,fg=structure.fg!75!black}
% \setbeamercolor{palette tertiary}{use=structure,fg=structure.fg!50!black}
% \setbeamercolor{palette quaternary}{fg=black}

% \setbeamercolor{local structure}{fg=primarycolor}
% \setbeamercolor{structure}{fg=primarycolor}
% \setbeamercolor{title}{fg=primarycolor}
% \setbeamercolor{section in head/foot}{fg=black}

% \setbeamercolor{normal text}{fg=black,bg=white}
% \setbeamercolor{block title alerted}{fg=red}
% \setbeamercolor{block title example}{fg=primarycolor}

% \setbeamercolor{footline}{fg=primarycolor!50}
% \setbeamerfont{footline}{series=\bfseries}

% use classic LaTeX font for maths
\usefonttheme[onlymath]{serif}

\usepackage{cmap}
\usepackage[english]{babel}
\usepackage[T1]{fontenc}
\usepackage[utf8]{inputenc}
\usepackage[kerning=true]{microtype}
\usepackage{lmodern}

\usepackage{amsmath}
\usepackage{amsfonts}
\usepackage{amssymb}
\usepackage{amsthm}

\usepackage{mathtools}
\usepackage{wrapfig}
% \usepackage{enumitem}
% \usepackage{tikz}
% \usepackage{xcolor}
% \usetikzlibrary{positioning}
\usepackage{booktabs}
\newcommand{\tabitem}{~~\llap{\textbullet}~~}

\usepackage[
    backend=biber,
    style=numeric,
]{biblatex}
\usepackage{graphicx}
\usepackage[justification=centering]{caption}
\usepackage{csquotes}
% \usepackage{multimedia}


\graphicspath{{./images/}}

\addbibresource{../report/report.bib}
% \renewcommand*{\bibfont}{\footnotesize}

\AtBeginSection[]
{
  \begin{frame}
    \frametitle{Plan}
    \tableofcontents[currentsection]
  \end{frame}
}


\theoremstyle{definition}
\newtheorem*{exemple}{Example}

\renewcommand{\leq}{\leqslant}
\renewcommand{\geq}{\geqslant}

% \setbeamertemplate{itemize items}[circle]


\title{\textbf{Implementation of an\\Iterative Linear Quadratic Regulator (iLQR)}}
\author{Gabriel Desfrene\and Antoine Groudiev}

\titlegraphic{\includegraphics[height=1.8cm]{./images/logo-ens-psl.png}}

\date{January 14, 2025}

\begin{document}
\frame{\titlepage}

\begin{frame}{Plan}
   \tableofcontents
\end{frame}

\section{Problem statement}
\begin{frame}{General formulation}
    \begin{itemize}
        \item Dynamics function:
        \begin{equation*}
            x_{t+1} = f(x_t, u_t)
        \end{equation*}
        \item Goal: minimize a quadratic cost function
        \item Cost function:
        \begin{equation*}
            J(u) = \sum_{t=0}^{T-1} \left(x_t^\top Q x_t + u_t^\top R u_t\right) + \frac{1}{2}(x_T-x^*)^\top Q_f (x_T-x^*)
        \end{equation*}
        \item $Q$: state cost matrix
        \item $Q_f$: final state cost matrix
        \item $R$: control cost matrix
    \end{itemize}
\end{frame}

\begin{frame}{Example: Simple Pendulum}
    \begin{minipage}[c]{0.6\linewidth}
        \begin{itemize}
            \item State: $x = [\theta\:\:\:\dot{\theta}]$
            \item Control: $u$, torque applied to the pendulum
            \item Dynamics: physical laws (simulator)
            \item Target: $x=[0\:\:\:0]$
            \item Cost function:
            \begin{equation*}
                J(u) = \frac{1}{2}\left(\theta_f^2+\dot{\theta}_f^2\right) + \frac{1}{2}\int_0^T ru^2(t) \mathrm{d} t
            \end{equation*}
            corresponding to $Q_f = I_2$, $Q=0_2$, $R=rI_1$
        \end{itemize}
    \end{minipage}
    \hspace{0.25cm}
    \begin{minipage}[c]{0.35\linewidth}
        \begin{figure}
            \centering
            \includegraphics[width=0.7\linewidth]{pendulum.png}
        \end{figure}
    \end{minipage}
\end{frame}

\begin{frame}{Example: Cartpole}
    \begin{minipage}[c]{0.6\linewidth}
        \begin{itemize}
            \item State: $x=[y\:\:\:\theta\:\:\:\dot{y}\:\:\:\dot{\theta}]$
            \item Control: $u$, force applied to the cart
            \item Dynamics: physical laws (simulator)
            \item Target: $x=[0\:\:\:0\:\:\:0\:\:\:0]$
            \item Cost function:
            \begin{equation*}
                J(u) = \frac{1}{2}\left(\theta_f^2+\dot{\theta}_f^2 + y_f^2 + \dot{y}_f^2\right) + \frac{1}{2}\int_0^T ru^2(t) \mathrm{d} t
            \end{equation*}
            corresponding to $Q_f = I_4$, $Q=0_4$, $R=rI_1$
        \end{itemize}
    \end{minipage}
    \hspace{0.25cm}
    \begin{minipage}[c]{0.35\linewidth}
        \begin{figure}
            \centering
            \includegraphics[width=0.7\linewidth]{cartpole.png}
        \end{figure}
    \end{minipage}
\end{frame}

\section{The iLQR algorithm}
\begin{frame}{General idea}
    \begin{itemize}
        \item iLQR is an iterative algorithm
        \item Start with an initial trajectory
        \item Iteratively improve it using a local linear approximation
        \item Stop when the trajectory converges
    \end{itemize}
\end{frame}

\begin{frame}{Linearizing the dynamics}
    The equation $x_{t+1} = f(x_t, u_t)$ is linearized (at each step) as:
    \begin{equation*}
        \delta x_{t+1} = A_t \delta x_t + B_t \delta u_t
    \end{equation*}
    with:
    \begin{itemize}
        \item $A_t$: Jacobian of $f$ with respect to $x$ evaluated at $(x_t, u_t)$
        \item $B_t$: Jacobian of $f$ with respect to $u$ evaluated at $(x_t, u_t)$
    \end{itemize}
    We are in LQR (Linear Quadratic Regulator, cf. TP5) setup!
\end{frame}

\begin{frame}{Trajectory refinement using LQR}
    \begin{enumerate}
        \item \textbf{Forward pass}: compute the successive states $(x_t)$ for the current controls $(u_t)$, and the corresponding cost $J$
        \item \textbf{Backward pass}: compute the gains, i.e. how much we should change the controls in each direction to minimize the cost
        \item \textbf{Forward rollout}: apply the gains to the controls to obtain a new trajectory
        \item Repeat until convergence
    \end{enumerate}
\end{frame}

\begin{frame}{Computing the Jacobians}{Finite differences method}
    We want to compute:
    \begin{itemize}
        \item $A_t=\frac{\partial f}{\partial x}(x_t, u_t)$, i.e. how much the state at time $t+1$ changes when we slightly change the state at time $t$
        \item $B_t=\frac{\partial f}{\partial u}(x_t, u_t)$, i.e. how much the state at time $t+1$ changes when we slightly change the control at time $t$
    \end{itemize}
    In a black box setting, we can use finite differences:
    \begin{align*}
        [A_t]_i &\approx \frac{f(x_t+\varepsilon e_i, u_t)-f(x_t-\varepsilon e_i, u_t)}{2\varepsilon}\\
        [B_t]_i &\approx \frac{f(x_t, u_t+\varepsilon e_i)-f(x_t, u_t-\varepsilon e_i)}{2\varepsilon}
    \end{align*}
    for some small $\varepsilon$ and the canonical basis $(e_i)$
\end{frame}

\begin{frame}{Computing the Jacobians}{Using Pinocchio}
    
\end{frame}

\section{Our implementation}
\begin{frame}{What language to use?}
    \begin{figure}
        \centering
        \renewcommand{\arraystretch}{1.4}
        \begin{tabular}{p{5cm}p{4cm}p{4cm}}
            \centering\textbf{Python}&\centering\textbf{C++}&\hspace{1.3cm}\textbf{Rust}\\
            \tabitem Easy to use&\tabitem Fast&\tabitem Fast\\
            \tabitem Bindings for many libraries&\tabitem Not very funny&\tabitem Very funny\\
            \tabitem Embarrassingly slow&&
        \end{tabular}
    \end{figure}
    % \begin{minipage}[c]{0.3\linewidth}
    %     \textbf{Python}
    %     \begin{itemize}
    %         \item Easy to use
    %         \item Bindings for many libraries
    %         \item Embarrassingly slow
    %     \end{itemize}
    % \end{minipage}
    % \hspace{0.25cm}
    % \begin{minipage}[c]{0.3\linewidth}
    %     \textbf{C++}
    %     \begin{itemize}
    %         \item Fast
    %         \item Not very funny
    %     \end{itemize}
    % \end{minipage}
    % \hspace{0.25cm}
    % \begin{minipage}[c]{0.3\linewidth}
    %     \textbf{Rust}
    % \end{minipage}
\end{frame}

\begin{frame}{From Rust to Python, and the other way around}
    
\end{frame}

\begin{frame}{API Basic usage}
    
\end{frame}

\section{Demonstration time}
\begin{frame}
    \begin{center}
        \huge Demonstration time!
    \end{center}
    % \begin{minipage}[c]{0.45\linewidth}
    %     \begin{figure}
    %         \centering
    %         \includegraphics[width=0.7\linewidth]{pendulum.png}
    %     \end{figure}
    % \end{minipage}
    % \hspace{0.25cm}
    % \begin{minipage}[c]{0.45\linewidth}
    %     \begin{figure}
    %         \centering
    %         \includegraphics[width=0.7\linewidth]{cartpole.png}
    %     \end{figure}
    % \end{minipage}
\end{frame}

% \begin{frame}
%     \movie{placeholder box}{images/balanced-pendulum.mov}
% \end{frame}


\section{Conclusion}
\begin{frame}[allowframebreaks]
    \nocite{*}
    \printbibliography
\end{frame}

\end{document}